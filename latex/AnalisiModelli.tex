\section{Analisi dei modelli}
In questa sezione si analizzeranno differenti modelli e successivamente li si confronteranno verificando quale dei modelli meglio soddisfa l'ipotesi di normalità dei residui tramite dei grafici e test diagnostici. Inoltre, dato il numero non elevato di campioni si confronteranno i valori di AIC e di adjusted-$R^2$.


\subsection{Modello 1}
Il primo modello analizzato è quello che include i regressori (di primo grado) più significativi (in base al valore di p\_value misurato precedentemente). Ovvero:
\begin{equation*}
y=\beta_0+\beta_1x_1+\beta_2x_2+\beta_3x_3+\beta_5x_5.
\end{equation*}
La stima dei parametri ottenuti per questo modello è
\begin{table}[H]
	\centering
	\begin{tabular}{|c|c|c|}
		\hline
		\textbf{Parametro} & \textbf{Stima} & \textbf{Dev. Std.} \\
		\hline
		$\beta_0$ & 65.62  & 1.30 \\
		$\beta_1$ & -9.37  & 1.38 \\
		$\beta_2$ & -13.33 & 1.24 \\
		$\beta_3$ & 4.01   & 1.26 \\
		$\beta_5$ & -14.52 & 1.26 \\
		\hline
	\end{tabular}
	\caption{Stime dei coefficienti e deviazioni standard del modello}
	\label{tab:coef_estimates}
\end{table}

Gli intervalli di confidenza al $5\%$, ottenuti tramite il metodo confint() di R, sono:
\begin{table}[H]
	\centering
	\begin{tabular}{|c|c|c|}
		\hline
		\textbf{Parametro} & \textbf{Lower bound} & \textbf{Upper bound} \\
		\hline
		$\beta_0$ & 63.04 & 68.20 \\
		$\beta_1$ & -12.11 & -6.62 \\
		$\beta_2$ & -15.79 & -10.87 \\
		$\beta_3$ & 1.51 & 6.51 \\
		$\beta_5$ & -17.01 & -12.03 \\
		\hline
	\end{tabular}
	\caption{Intervalli di confidenza al 95\% per i coefficienti del modello}
	\label{tab:ci_coefficienti}
\end{table}

I valori dell'adjusted $R^2$  e AIC ottenuti sono:
\begin{equation*}
	R^2 =  0.77, \quad AIC = 514.69.
\end{equation*}
\subsection{Modello 2}
Il prossimo modello analizzato è quello ottenuto aggiungendo tutti i regressori più significativi con l'aggiunta di alcuni regressori al quadrato.
\begin{equation*}
	y=\beta_0 + \beta_1x_1+\beta_2x_1^2+\beta_3x_2+\beta_4x_2^2+\beta_5x_3+\beta_6x_5
\end{equation*}
La stima dei parametri ottenuti per questo modello è
\begin{table}[H]
	\centering
	\begin{tabular}{|c|c|c|}
		\hline
		\textbf{Parametro} & \textbf{Stima} & \textbf{Dev. Std.} \\
		\hline
		$\beta_0$ & 79.93  & 1.95 \\
		$\beta_1$ & -8.66  & 1.05 \\
		$\beta_2$ & -8.03  & 1.23 \\
		$\beta_3$ & -13.49 & 0.94 \\
		$\beta_4$ & -6.38  & 1.09 \\
		$\beta_5$ & 3.94   & 0.95 \\
		$\beta_6$ & -13.23 & 0.96 \\
		\hline
	\end{tabular}
	\caption{Stime dei coefficienti e errori standard del modello}
	\label{tab:coef_estimates_poly}
\end{table}

Gli intervalli di confidenza al $5\%$, ottenuti tramite il metodo confint() di R, sono:
\begin{table}[H]
	\centering
	\begin{tabular}{|c|c|c|}
		\hline
		\textbf{Parametro} & \textbf{Lower bound} & \textbf{Upper bound} \\
		\hline
		$\beta_0$ & 76.06 & 83.80 \\
		$\beta_1$ & -10.75 & -6.58 \\
		$\beta_2$ & -10.48 & -5.58 \\
		$\beta_3$ & -15.36 & -11.63 \\
		$\beta_4$ & -8.55 & -4.22 \\
		$\beta_5$ & 2.05 & 5.84 \\
		$\beta_6$ & -15.14 & -11.32 \\
		\hline
	\end{tabular}
	\caption{Intervalli di confidenza al 95\% per i coefficienti del modello}
	\label{tab:ci_coefficienti}
\end{table}

I valori dell'adjusted $R^2$  e AIC ottenuti sono:
\begin{equation*}
	R^2 =   0.87, \quad AIC = 460.76.
\end{equation*}

\subsection{Modello 3}
Questo modello è ottenuto tramite la seguente istruzione R, adottando la funzione step():
\begin{align*}
	y &= \beta_0 
	+ \beta_1 x_1 + \beta_2 x_2 + \beta_3 x_3 + \beta_4 x_4 + \beta_5 x_5 
	+ \beta_6 x_6 + \beta_7 x_7 
	+ \beta_8 x_1^2 + \beta_9 x_2^2 + \beta_{10} x_6^2 + \beta_{11} x_7^2 \\
	&+ \beta_{12} x_1 x_6 + \beta_{13} x_2 x_4 + \beta_{14} x_3 x_4 
	+ \beta_{15} x_3 x_5 + \beta_{16} x_3 x_7 
	+ \beta_{17} x_4 x_7.
\end{align*}
In particolare il modello di partenza da cui si è partiti:
\begin{verbatim}
	model_step_interactions <- lm(y_VideoQuality ~ (.)^2 + I(x1_ISO^2) + 
	I(x2_FRatio^2) + I(x3_TIME^2) + I(x4_MP^2) + I(x5_CROP^2) + I(x6_FOCAL^2) + 
	I(x7_PixDensity^2), data = data)
\end{verbatim}
La stima dei parametri ottenuti per questo modello è:
\begin{table}[H]
	\centering
	\begin{tabular}{|c|c|c|}
		\hline
		\textbf{Parametro} & \textbf{Stima} & \textbf{Dev. Std.} \\
		\hline
		$\beta_0$   & 81.64  & 2.18 \\
		$\beta_1$   & -8.77  & 1.00 \\
		$\beta_2$   & -13.56 & 0.90 \\
		$\beta_3$   & 4.31   & 1.03 \\
		$\beta_4$   & -0.25  & 1.46 \\
		$\beta_5$   & -13.37 & 0.92 \\
		$\beta_6$   & 0.62   & 0.99 \\
		$\beta_7$   & -2.96  & 1.60 \\
		$\beta_8$   & -8.85  & 1.16 \\
		$\beta_9$   & -6.57  & 1.01 \\
		$\beta_{10}$ & -1.89  & 1.07 \\
		$\beta_{11}$ & 2.91   & 1.86 \\
		$\beta_{12}$ & -1.71  & 1.18 \\
		$\beta_{13}$ & 1.66   & 0.99 \\
		$\beta_{14}$ & -2.81  & 1.42 \\
		$\beta_{15}$ & 2.83   & 0.99 \\
		$\beta_{16}$ & 3.24   & 1.54 \\
		$\beta_{17}$ & -3.55  & 2.25 \\
		\hline
	\end{tabular}
	\caption{Stime dei coefficienti e deviazioni standard del modello}
	\label{tab:stima_coef_std}
\end{table}
Gli intervalli di confidenza al $5\%$, ottenuti tramite il metodo confint() di R, sono:
\begin{table}[H]
	\centering
	\begin{tabular}{|c|c|c|}
		\hline
		\textbf{Parametro} & \textbf{Lower bound} & \textbf{Upper bound} \\
		\hline
		$\beta_0$   & 77.29  & 85.99 \\
		$\beta_1$   & -10.76 & -6.78 \\
		$\beta_2$   & -15.34 & -11.77 \\
		$\beta_3$   & 2.26   & 6.37 \\
		$\beta_4$   & -3.16  & 2.65 \\
		$\beta_5$   & -15.20 & -11.53 \\
		$\beta_6$   & -1.34  & 2.59 \\
		$\beta_7$   & -6.14  & 0.22 \\
		$\beta_8$   & -11.16 & -6.55 \\
		$\beta_9$   & -8.58  & -4.57 \\
		$\beta_{10}$ & -4.01  & 0.23 \\
		$\beta_{11}$ & -0.78  & 6.61 \\
		$\beta_{12}$ & -4.05  & 0.64 \\
		$\beta_{13}$ & -0.31  & 3.62 \\
		$\beta_{14}$ & -5.65  & 0.02 \\
		$\beta_{15}$ & 0.86   & 4.81 \\
		$\beta_{16}$ & 0.19   & 6.30 \\
		$\beta_{17}$ & -8.02  & 0.93 \\
		\hline
	\end{tabular}
	\caption{Intervalli di confidenza al 95\% per i coefficienti del modello}
	\label{tab:ci_coefficienti}
\end{table}
I valori dell'adjusted $R^2$  e AIC ottenuti sono:
\begin{equation*}
	R^2 =      0.89, \quad AIC=448.27.
\end{equation*}

\subsection{Modello 4}
Questo modello è stato ottenuto riducendo il Modello 3, in particolare escludendo le variabili $x_4$ e $x_6$ apparentemente poco significative. Si presenta come:

\begin{align*}
	y = \beta_0 + \beta_1x_1 + \beta_2x_2 + \beta_3x_3 + \beta_4x_5 + \beta_5x_7 + \beta_6x_1^2 + \beta_7x_2^2 + \beta_8x_3x_5
\end{align*}


La stima dei parametri ottenuti per questo modello è:
\begin{table}[H]
	\centering
	\begin{tabular}{|c|c|c|}
		\hline
		\textbf{Parametro} & \textbf{Stima} & \textbf{Dev. Std.} \\
		\hline
		$\beta_0$   & 80.00  & 1.86 \\
		$\beta_1$   & -8.19  & 1.00 \\
		$\beta_2$   & -13.59 & 0.90 \\
		$\beta_3$   & 4.73   & 0.96 \\
		$\beta_4$   & -13.22  & 0.92 \\
		$\beta_5$   & -2.91 & 0.94 \\
		$\beta_6$   & -8.31   & 1.18 \\
		$\beta_7$   & -6.26  & 1.04 \\
		$\beta_8$   & 1.89  & 0.94 \\
		\hline
	\end{tabular}
	\caption{Stime dei coefficienti e deviazioni standard del modello}
	\label{tab:stima_coef_std}
\end{table}
Gli intervalli di confidenza al $5\%$, ottenuti tramite il metodo confint() di R, sono:
\begin{table}[H]
	\centering
	\begin{tabular}{|c|c|c|}
		\hline
		\textbf{Parametro} & \textbf{Lower bound} & \textbf{Upper bound} \\
		\hline
		$\beta_0$   & 76.31  & 83.69 \\
		$\beta_1$   & -10.18 & -6.19 \\
		$\beta_2$   & -15.38 & -11.80 \\
		$\beta_3$   & 2.82   & 6.64 \\
		$\beta_4$   & -15.04  & -11.39 \\
		$\beta_5$   & -4.77 & -1.05 \\
		$\beta_6$   & -10.65  & -5.97 \\
		$\beta_7$   & -8.32  & -4.21 \\
		$\beta_8$   & 0.02 & 3.75 \\
		\hline
	\end{tabular}
	\caption{Intervalli di confidenza al 95\% per i coefficienti del modello}
	\label{tab:ci_coefficienti}
\end{table}
I valori dell'adjusted $R^2$  e AIC ottenuti sono:
\begin{equation*}
	R^2 =      0.88, \quad AIC=451.56.
\end{equation*}


\subsection{Modello 5}
Questo modello è stato ottenuto analizzando anche i termini cubici. In particolare, il modello si presenta nel seguente modo:

\begin{align*}
	y &= \beta_0 + \beta_1x_1 + \beta_2x_2 + \beta_3x_3 + \beta_4x_5 + \beta_5x_6 + \beta_6x_7 + \beta_7x_1^2 + \beta_8x_2^2 + \beta_9x_6^2 + \beta_{10}x_1^3 + \beta_{11}x_7^3 \\
	&+ \beta_{12}x_1x_7 + \beta_{13}x_3x_5 + \beta_{14}x_3x_7
\end{align*}

La stima dei parametri ottenuti per questo modello è:
\begin{table}[H]
	\centering
	\begin{tabular}{|c|c|c|}
		\hline
		\textbf{Parametro} & \textbf{Stima} & \textbf{Dev. Std.} \\
		\hline
		$\beta_0$   & 81.87  & 1.97 \\
		$\beta_1$   & -0.44  & 2.16 \\
		$\beta_2$   & -13.46 & 0.84 \\
		$\beta_3$   & 4.61   & 0.90 \\
		$\beta_4$   & -13.74  & 0.83 \\
		$\beta_5$   & 1.57 & 0.92 \\
		$\beta_6$   & -6.08   & 1.76 \\
		$\beta_7$   & -8.63  & 1.10 \\
		$\beta_8$   & -6.79  & 0.93 \\
		$\beta_9$   & -1.80  & 0.97 \\
		$\beta_{10}$ & -4.99  & 1.27 \\
		$\beta_{11}$ & 1.97   & 0.74 \\
		$\beta_{12}$ & 1.59  & 1.00 \\
		$\beta_{13}$ & 1.41   & 0.87 \\
		$\beta_{14}$ & 1.78  & 0.86 \\
		\hline
	\end{tabular}
	\caption{Stime dei coefficienti e deviazioni standard del modello}
	\label{tab:stima_coef_std}
\end{table}
Gli intervalli di confidenza al $5\%$, ottenuti tramite il metodo confint() di R, sono:
\begin{table}[H]
	\centering
	\begin{tabular}{|c|c|c|}
		\hline
		\textbf{Parametro} & \textbf{Lower bound} & \textbf{Upper bound} \\
		\hline
		$\beta_0$   & 77.95  & 85.80 \\
		$\beta_1$   & -4.73 & 3.86 \\
		$\beta_2$   & -15.13 & -11.80 \\
		$\beta_3$   & 2.81   & 6.40 \\
		$\beta_4$   & -15.39  & -12.08 \\
		$\beta_5$   & -0.25 & -3.39 \\
		$\beta_6$   & -9.58  & -2.57 \\
		$\beta_7$   & -10.82  & -6.44 \\
		$\beta_8$   & -8.63 & -4.95 \\
		$\beta_9$   & -3.72  & 0.12 \\
		$\beta_{10}$ & -7.52  & -2.47 \\
		$\beta_{11}$ & 0.50  & 3.43 \\
		$\beta_{12}$ & -0.39  & 3.58 \\
		$\beta_{13}$ & -0.31  & 3.14 \\
		$\beta_{14}$ & 0.07  & 3.49 \\
		\hline
	\end{tabular}
	\caption{Intervalli di confidenza al 95\% per i coefficienti del modello}
	\label{tab:ci_coefficienti}
\end{table}
I valori dell'adjusted $R^2$  e AIC ottenuti sono:
\begin{equation*}
	R^2 =      0.92, \quad AIC=431.91.
\end{equation*}