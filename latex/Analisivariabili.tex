\section{Analisi della dipendenza tra le variabili}

\subsection{Analisi di correlazione}

\begin{figure}[H]
	\centering
	\includegraphics[width=0.90\textwidth]{../graphs/DescriptiveStatisticPlots/ggplot}
	\caption{Scatter plot delle variabili presenti nel dataset.}
\end{figure}

Dalla Figura (3) notiamo, anche dal coefficiente di correlazione, una dipendenza lineare tra le variabili:
\begin{itemize}
	\item \textbf{x4\_MP} e \textbf{x7\_PixDensity}
\end{itemize}
Invece notiamo la presenza di dipendenze non lineari che non vengono descritte dal coefficiente di correlazione. In particolare la notiamo tra le variabili:
\begin{itemize}
	\item \textbf{y\_VideoQuality} e \textbf{x1\_ISO}
	\item \textbf{y\_VideoQuality} e \textbf{x2\_FRatio}
	\item \textbf{y\_VideoQuality} e \textbf{x3\_Time}
	\item \textbf{y\_VideoQuality} e \textbf{x5\_CROP}
	\item \textbf{x5\_CROP} e \textbf{x7\_PixDensity}
\end{itemize}

\subsection{Analisi di regressione}

Le dipendenze tra la variabile y\_VideoQuality e le diverse variabili indipendenti sono state analizzate attraverso una regressione polinomiale.

\begin{table}[H]
	\centering
	\begin{tabular}{|c|c|}
		\hline
		\textbf{Variabile indipendente} & \textbf{p-value} \\
		\hline
		x1\_ISO & $1.17e-05$ \\
		\hline
	\end{tabular}
	\caption{}
	\label{tab:}
\end{table}





