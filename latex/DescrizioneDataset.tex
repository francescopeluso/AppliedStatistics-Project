\section{Descrizione del dataset fornito}
A completezza del progetto si riporta la descrizione del dataset da analizzare.

\subsection*{Variabile dipendente}
\textbf{y\_VideoQuality} \quad $\rightarrow$ \quad Qualità percepita del video

Tale indice è immaginato come frutto di una opportuna trasformazione di un punteggio assegnato a un campione di immagini da volontari che compilano un questionario. Esso sarà funzione di diverse caratteristiche proprie dei video, tra cui:

\begin{itemize}
	\item la presenza o meno di rumore;
	\item la presenza o meno di \textit{motion blur};
	\item la nitidezza;
	\item la profondità di campo;
	\item la risoluzione;
	\item le aberrazioni ottiche visibili;
	\item la gamma dinamica;
	\item la fedeltà cromatica.
\end{itemize}

\subsection*{Variabili indipendenti (regressori)}

Sono delle quantità di cui l’operatore ha il controllo (parziale o totale) selezionando:

\begin{itemize}
	\item l’attrezzatura video da utilizzare;
	\item i parametri di ripresa.
\end{itemize}

Rappresentano indici standardizzati:

\begin{itemize}
	\item \texttt{x1\_ISO} \quad $\rightarrow$ \quad ISO (sensibilità del sensore)
	\item \texttt{x2\_FRatio} \quad $\rightarrow$ \quad Rapporto Focale
	\item \texttt{x3\_Time} \quad $\rightarrow$ \quad Tempo di Esposizione (in relazione al frame rate utilizzato)
	\item \texttt{x4\_MP} \quad $\rightarrow$ \quad Megapixel del sensore
	\item \texttt{x5\_CROP} \quad $\rightarrow$ \quad Fattore di Crop
	\item \texttt{x6\_FOCAL} \quad $\rightarrow$ \quad Focale
	\item \texttt{x7\_PixDensity} \quad $\rightarrow$ \quad Densità di pixel
\end{itemize}


