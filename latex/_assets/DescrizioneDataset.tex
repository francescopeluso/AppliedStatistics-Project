\section{Introduzione}

\subsection{Descrizione del progetto}
Il progetto si propone di sviluppare un modello di regressione lineare multipla per analizzare la qualità percepita dei video in funzione di diverse caratteristiche tecniche.  
L'obiettivo è comprendere come tali caratteristiche influenzino il giudizio soggettivo degli utenti, espresso tramite un punteggio assegnato ai contenuti audiovisivi.

\subsection{Descrizione del dataset}
Il dataset considerato è costituito da $n = 100$ osservazioni, ciascuna delle quali rappresenta un video valutato da un gruppo di volontari. Le variabili presenti nel dataset si suddividono in:

\subsubsection*{Variabile dipendente (target)}
\textbf{y\_VideoQuality} \quad $\rightarrow$ \quad Qualità percepita del video

\noindent Questo indice è ottenuto da un'opportuna trasformazione dei punteggi soggettivi raccolti tramite questionari. Esso tiene conto di fattori percettivi quali:

\begin{itemize}
    \item presenza di rumore;
    \item presenza di \textit{motion blur};
    \item nitidezza;
    \item profondità di campo;
    \item risoluzione;
    \item aberrazioni ottiche visibili;
    \item gamma dinamica;
    \item fedeltà cromatica.
\end{itemize}

\subsubsection*{Variabili indipendenti (regressori)}

Le variabili esplicative sono costituite da indici standardizzati relativi ai parametri di acquisizione, controllabili in fase di ripresa attraverso la scelta dell'attrezzatura e delle impostazioni operative. Le variabili considerate sono:

\begin{itemize}
    \item \textbf{x1\_ISO} \quad $\rightarrow$ \quad Sensibilità ISO del sensore
    \item \textbf{x2\_FRatio} \quad $\rightarrow$ \quad Rapporto focale
    \item \textbf{x3\_Time} \quad $\rightarrow$ \quad Tempo di esposizione (relativo al frame rate)
    \item \textbf{x4\_MP} \quad $\rightarrow$ \quad Numero di megapixel del sensore
    \item \textbf{x5\_CROP} \quad $\rightarrow$ \quad Fattore di crop
    \item \textbf{x6\_FOCAL} \quad $\rightarrow$ \quad Lunghezza focale
    \item \textbf{x7\_PixDensity} \quad $\rightarrow$ \quad Densità di pixel
\end{itemize}

\noindent Nei capitoli seguenti verrà illustrato il processo di modellazione statistica adottato, includendo le fasi di analisi descrittiva, costruzione dei modelli, valutazione comparativa e diagnostica.
